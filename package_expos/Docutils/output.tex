\documentclass[a4paper]{article}
% generated by Docutils <http://docutils.sourceforge.net/>
\usepackage{cmap} % fix search and cut-and-paste in Acrobat
\usepackage{ifthen}
\usepackage[T1]{fontenc}
\usepackage[utf8]{inputenc}
\usepackage{alltt}
\setcounter{secnumdepth}{0}

%%% Custom LaTeX preamble
% PDF Standard Fonts
\usepackage{mathptmx} % Times
\usepackage[scaled=.90]{helvet}
\usepackage{courier}

%%% User specified packages and stylesheets

%%% Fallback definitions for Docutils-specific commands

% hyperlinks:
\ifthenelse{\isundefined{\hypersetup}}{
  \usepackage[colorlinks=true,linkcolor=blue,urlcolor=blue]{hyperref}
  \usepackage{bookmark}
  \urlstyle{same} % normal text font (alternatives: tt, rm, sf)
}{}
\hypersetup{
  pdftitle={rst2html5 tools - RestructuredText to HTML5 + bootstrap css},
}

%%% Body
\begin{document}
\title{rst2html5 tools - RestructuredText to HTML5 + bootstrap css%
  \label{rst2html5-tools-restructuredtext-to-html5-bootstrap-css}}
\author{}
\date{}
\maketitle

We all love rst and the ability to generate any format, but the rst2html tool generates really basic html and css.

This tool will generate newer, nicer, more readable markup and provide ways to modify the output with extensions like nice css
thanks to twitter's bootstrap css or online presentations with deck.js


\section{get it%
  \label{get-it}%
}

via pip:

\begin{quote}
\begin{alltt}
pip install rst2html5-tools
\end{alltt}
\end{quote}

locally:

\begin{quote}
\begin{alltt}
git clone https://github.com/marianoguerra/rst2html5.git
cd rst2html5
git submodule init
git submodule update

sudo python setup.py install
\end{alltt}
\end{quote}


\section{use it%
  \label{use-it}%
}

to generate a basic html document:

\begin{quote}
\begin{alltt}
rst2html5 examples/slides.rst > clean.html
\end{alltt}
\end{quote}

to generate a set of slides using deck.js:

\begin{quote}
\begin{alltt}
rst2html5 -{}-deck-js -{}-pretty-print-code -{}-embed-content examples/slides.rst > deck.html
\end{alltt}
\end{quote}

to generate a set of slides using reveal.js:

\begin{quote}
\begin{alltt}
rst2html5 -{}-jquery -{}-reveal-js -{}-pretty-print-code examples/slides.rst > reveal.html
\end{alltt}
\end{quote}

to generate a set of slides using impress.js:

\begin{quote}
\begin{alltt}
rst2html5 -{}-stylesheet-path=html5css3/thirdparty/impressjs/css/impress-demo.css -{}-impress-js examples/impress.rst > output/impress.html
\end{alltt}
\end{quote}

to generate a page using bootstrap:

\begin{quote}
\begin{alltt}
rst2html5 -{}-bootstrap-css -{}-pretty-print-code -{}-jquery -{}-embed-content examples/slides.rst > bootstrap.html
\end{alltt}
\end{quote}

to higlight code with pygments:

\begin{quote}
\begin{alltt}
rst2html5 -{}-pygments examples/codeblock.rst > code.html
\end{alltt}
\end{quote}

note that you will have to add the stylesheet for the code to actually highlight, this just does the code parsing and html transformation.

to embed images inside the html file to have a single .html file to distribute
add the -{}-embed-images option.

post processors support optional parameters, they are passed with a command
line option with the same name as the post processor appending \textquotedbl{}-opts\textquotedbl{} at the
end, for example to change the revealjs theme you can do:

\begin{quote}
\begin{alltt}
rst2html5 -{}-jquery -{}-reveal-js -{}-reveal-js-opts theme=serif examples/slides.rst > reveal.html
\end{alltt}
\end{quote}

you can also pass the base path to the theme css file:

\begin{quote}
\begin{alltt}
rst2html5 -{}-jquery -{}-reveal-js -{}-reveal-js-opts theme=serif,themepath=~/mytheme examples/slides.rst > reveal.html
\end{alltt}
\end{quote}

it will look at the theme at \textasciitilde{}/mytheme/serif.css

options are passed as a comma separated list of key value pairs separated with
an equal sign, values are parsed as json, if parsing fails they are passed as
strings, for example here is an example of options:

\begin{quote}
\begin{alltt}
-{}-some-processor-opts theme=serif,count=4,verbose=true,foo=null
\end{alltt}
\end{quote}

if a key is passed more than once that parameter is passed to the processor as a list of values, note that if only one value is passed it's passed as it is, the convenience function as\_list is provided to handle this case if you want to always receive a list.

to add custom js files to the resulting file you can use the -{}-add-js post processor like this:

\begin{quote}
\begin{alltt}
rst2html5 slides.rst -{}-add-js -{}-add-js-opts path=foo.js,path=bar.js
\end{alltt}
\end{quote}

that command will add foo.js and bar.js as scripts in the resulting html file.


\subsection{Pretty Print Code Notes%
  \label{pretty-print-code-notes}%
}

enable it:

\begin{quote}
\begin{alltt}
-{}-pretty-print-code
\end{alltt}
\end{quote}

add language specific lexers:

\begin{quote}
\begin{alltt}
-{}-pretty-print-code-opts langs=clj:erlang
\end{alltt}
\end{quote}

Note: you have to pass both options when passing opts to prettify like this:

\begin{quote}
\begin{alltt}
-{}-pretty-print-code -{}-pretty-print-code-opts langs=clj:erlang
\end{alltt}
\end{quote}

that is, the name of the languages separated by colons, available lexers at the
moment of this writing are:

\begin{itemize}
\item apollo

\item basic

\item clj

\item css

\item dart

\item erlang

\item go

\item hs

\item lisp

\item llvm

\item lua

\item matlab

\item ml

\item mumps

\item n

\item pascal

\item proto

\item rd

\item r

\item scala

\item sql

\item tcl

\item tex

\item vb

\item vhdl

\item wiki

\item xq

\item yaml
\end{itemize}

you can see the available lexers under html5css3/thirdparty/prettify/lang-*.js


\subsection{RevealJs Notes%
  \label{revealjs-notes}%
}

to print pass -{}-reveal-js-opts printpdf=true, for example:

\begin{quote}
\begin{alltt}
rst2html5 -{}-jquery -{}-reveal-js -{}-reveal-js-opts printpdf=true examples/slides.rst > reveal-print.html
\end{alltt}
\end{quote}

this can be used to open with chrome or chromium and print as pdf as described here: \url{https://github.com/hakimel/reveal.js\#pdf-export}


\section{Math Support%
  \label{math-support}%
}

Use the \texttt{math} role and directive to include inline math and block-level equations into your document:

\begin{quote}
\begin{alltt}
When :math`a \textbackslash{}ne 0`, there are two solutions to :math:`ax^2 + bx + c = 0`
and they are

.. math::

   x = \{-b \textbackslash{}pm \textbackslash{}sqrt\{b^2-4ac\} \textbackslash{}over 2a\}
\end{alltt}
\end{quote}

Both of these support a basic subset of \href{https://www.latex-project.org}{LaTeX} syntax.

By default, \href{https://www.mathjax.org}{MathJax} is used for displaying math. You can choose a different output format using the \texttt{-{}-math-output} command line option:

\begin{itemize}
\item \texttt{-{}-math-output mathjax} uses MathJax (the default)

\item \texttt{-{}-math-output html} will use plain HTML + CSS

\item \texttt{-{}-math-output mathml} will use \href{https://en.wikipedia.org/wiki/MathML}{MathML}

\item \texttt{-{}-math-output latex} outputs raw LaTeX
\end{itemize}

If you use MathJax, you can use the \texttt{-{}-mathjax-url} and \texttt{-mathjax-config} command line options to configure a custom MathJax JavaScript URL and to provide a file with a custom MathJax configuration, respectively.

If you use HTML + CSS output, you can use the \texttt{-{}-math-css} command line option to configure a custom math stylesheet.

Note that the old MathJax postprocessor (activated using \texttt{-{}-mathjax}) has been deprecated.


\section{see it%
  \label{see-it}%
}

you can see the examples from the above commands here:

\begin{itemize}
\item \url{http://marianoguerra.github.com/rst2html5/output/clean.html}

\item \url{http://marianoguerra.github.com/rst2html5/output/reveal.html}

\item \url{http://marianoguerra.github.com/rst2html5/output/deck.html}

\item \url{http://marianoguerra.github.com/rst2html5/output/impress.html}

\item \url{http://marianoguerra.github.com/rst2html5/output/bootstrap.html}
\end{itemize}

example of video directive

\begin{itemize}
\item \url{http://marianoguerra.github.com/rst2html5/output/videos.html}
\end{itemize}


\section{test it%
  \label{test-it}%
}

We use \href{https://tox.readthedocs.org}{tox} to run our test suite. After installing \emph{tox} you can execute the tests by running \texttt{tox} in the project's root directory.

The test cases can be found in \texttt{html5css3/tests.py}.


\section{want to contribute ?%
  \label{want-to-contribute}%
}

clone and send us a pull request!

\begin{quote}
\begin{alltt}
git clone https://github.com/marianoguerra/rst2html5.git
cd rst2html5
git submodule update -{}-init
python setup.py develop
\end{alltt}
\end{quote}


\section{note to self to release%
  \label{note-to-self-to-release}%
}

\begin{itemize}
\item update version on setup.py
\end{itemize}

\begin{quote}
\begin{alltt}
python setup.py sdist upload
\end{alltt}
\end{quote}

\end{document}
